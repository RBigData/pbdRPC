\section[Handling Machine Information]{Handling Machine Information}
\label{sec:handling_machine_information}
\addcontentsline{toc}{section}{\thesection. Handling Machine Information}

In Section~\ref{sec:handling_login_information}, we have seen a very
tedious way to handle login information which also includes some
information for a single machine. In this section, we introduce a better way
to handle both login information and multiple machines.
The function \code{machine()} will generate a constructor-like object
containing all required information.
It is as simple as the example below.
\begin{Code}[title=Set machine information]
> library(pbdRPC, quietly = TRUE)
>
> ### Multiple machine information as needed
> m1 <- machine(user = "snoweye", hostname = "192.168.56.101", pport = 22)
> m2 <- machine(user = "snoweye", hostname = "192.168.56.102", pport = 22)
> m3 <- machine(user = "snoweye", hostname = "192.168.56.103", pport = 22)
\end{Code}

With the above objects \code{m1}, \code{m2}, and \code{m3}, the function
\code{rpc()} can assess freely to three machines with simpler
interface then the function \code{srpc()}.
For example, one may quickly check the access of three machines as the
example below.
\begin{Code}[title=Basic \code{rpc()} in \pkg{pbdRPC} and \proglang{R}]
> rpc(m1, "uname")
> rpc(m2, "uname")
> rpc(m3, "uname")
\end{Code}

