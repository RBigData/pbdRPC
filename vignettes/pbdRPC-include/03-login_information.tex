\section[Handling Login Information]{Handling Login Information}
\label{sec:handling_login_information}
\addcontentsline{toc}{section}{\thesection. Handling Login Information}

Suppose an Oracle VM VirtualBox runs Xubuntu 15.10 as the guest OS within a
Windows 8 system as the host OS.
The VM has an virtual network adaptor (host-only)
with IP address \code{192.168.56.101}, so that one can
login to the VM using either \code{telnet}, \code{plink}, or \code{ssh}
from the Windows 8 system.
Note that \code{telnet} and \code{ssh} uses ports 23 and 22 as default,
respectively.
Suppose further the login id is called ``snoweye'', then one may use
the function \code{rpcopt_set()} to assign/overwrite the login information to
\code{.pbd_env$RPC.LI} as in next.
\begin{Code}[title=Set login information]
> ### Alter login information as needed
> rpcopt_set(user = "snoweye", hostname = "192.168.56.101", pport = "22")
\end{Code}

In next, the basic login information \code{RPC.LI} describes that
\code{rpc()} will
\begin{itemize}
\item use \code{ssh} (exec.type) to execute a command
      (given by \code{rpc()}, \code{ssh()}, or \code{plink()})
\item with \code{args} (additional arguments to \code{ssh} or \code{plink.exe})
\item and a \code{user} account (snoweye)
\item login into a \code{hostname} (server ip = \code{192.168.56.101} or
      host name)
\item from a \code{pport} (server port = 22), and
\item may use authentication keys in \code{priv.key} or \code{priv.key.ppk}.
\end{itemize}
\begin{Code}[title=Basic \code{RPC.LI}]
> .pbd_env$RPC.LI
$exec.type
[1] "ssh"

$args
[1] ""

$pport
[1] "22"

$user
[1] "snoweye"

$hostname
[1] "192.168.56.101"

$priv.key
[1] "~/.ssh/id_rsa"

$priv.key.ppk
[1] "./id_rsa.ppk"
\end{Code}
Currently, the \code{exec.type} is only for non-Windows systems, and it
will be ignored on Windows systems ("plink" will be used).
Also, \code{ssh} uses ``\code{-p}'' (lower case) to input the server port
argument.
\code{plink.exe} uses ``\code{-P}'' (upper case) to input the server port
argument.
Therefore, the \code{args} should not include ``\code{-p}'' nor ``\code{-P}''
to avoid confusion in the unified function \code{rpc()}.
Similarly, the ``\code{-i}'' may not be include in the \code{args} as well
because additional authentication may be required.

The account may have the private key for authentication to avoid typing
the login password for the user account.
The private keys may be stored in files indicated by
\code{prive.key} for \code{ssh()} or
\code{prive.key.ppk} for \code{plink()}.
When all setups are correct, command calls can be executed at the
\code{hostname} (192.168.56.101) remotely.
By default, the \code{prive.key.ppk} will be read from the current
working directory (from \code{getwd()}) in Windows systems.
In this case (\code{"./id_rsa.ppk"}), the file
\code{C:/Users/login_account/Documents/id_rsa.ppk} is probably read
for authentication.

To generate private and public keys
is pretty standard for most Linux systems via the \code{ssh-keygen} command
which will generate keys in OpenSSH format.
One may use \code{puttygen} in Linux to convert OpenSSH format to
PuTTY format for Windows.
See Section~\ref{sec:general} for the conversion from \code{id_rsa} to
\code{id_rsa.ppk}.
For Windows systems, one may also use \code{puttygen.exe} to obtain both keys.

